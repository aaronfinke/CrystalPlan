%------------------------------------------------------------------------------
% Template file for the submission of papers to IUCr journals in LaTeX2e
% using the iucr document class
% Copyright 1999-2009 International Union of Crystallography
% Version 1.4 (11 May 2009)
%------------------------------------------------------------------------------

\documentclass{iucr}              % DO NOT DELETE THIS LINE

     %-------------------------------------------------------------------------
     % Information about the type of paper
     %-------------------------------------------------------------------------
     \paperprodcode{a000000}      % Replace with production code if known
     \paperref{xx9999}            % Replace xx9999 with reference code if known
     \papertype{FA}               % Indicate type of article
                                  %   FA - research papers (full article)
                                  %   SC - short communications
                                  %   LA - lead article
                                  %   FE - feature articles
                                  %   ST - structural communications
                                  %   XC - crystallization communications
                                  % (Following categories rarely in LaTeX)
                                  %   AA - abstracts
                                  %   AD - addenda and errata
                                  %   BC - books received
                                  %   BR - book reviews
                                  %   CA - cif applications
                                  %   CE - current events
                                  %   CI - inorganic compounds
                                  %   CM - metal-organic compounds
                                  %   CN - cryocrystallography papers
                                  %   CO - organic compounds
                                  %   CP - computer programs
                                  %   CR - crystallographers
                                  %   CS - scientific comment
                                  %   ED - editorial
                                  %   EI - inorganic compounds
                                  %   EM - metal-organic compounds
                                  %   EO - organic compounds
                                  %   FI - inorganic compounds
                                  %   FM - metal-organic compounds
                                  %   FO - organic compounds
                                  %   IP - issue preface
                                  %   IU - iucr
                                  %   LE - letters to the editor
                                  %   LN - laboratory notes
                                  %   ME - forthcoming meetings/short courses
                                  %   MR - meeting reports
                                  %   NN - notes and news
                                  %   NP - new commercial products
                                  %   OB - obituaries
                                  %   SR - software reviews
                                  %   TE - teaching and education

     \paperlang{english}          % Can be english, french, german or russian
     %-------------------------------------------------------------------------
     % Information about journal to which submitted
     %-------------------------------------------------------------------------
     \journalcode{J}              % Indicate the journal to which submitted
                                  %   A - Acta Crystallographica Section A
                                  %   B - Acta Crystallographica Section B
                                  %   C - Acta Crystallographica Section C
                                  %   D - Acta Crystallographica Section D
                                  %   E - Acta Crystallographica Section E
                                  %   F - Acta Crystallographica Section F
                                  %   J - Journal of Applied Crystallography
                                  %   S - Journal of Synchrotron Radiation
          %--------------------------------------------------------------------
          % The following entries will be changed as required by editorial staff
          %--------------------------------------------------------------------
     \journalyr{2010}
     \journaliss{1}
     \journalvol{65}
     \journalfirstpage{000}
     \journallastpage{000}
     \journalreceived{0 XXXXXXX 0000}
     \journalaccepted{0 XXXXXXX 0000}
     \journalonline{0 XXXXXXX 0000}

\begin{document}                  % DO NOT DELETE THIS LINE

     %-------------------------------------------------------------------------
     % The introductory (header) part of the paper
     %-------------------------------------------------------------------------

     % The title of the paper. Use \shorttitle to indicate an abbreviated title
     % for use in running heads (you will need to uncomment it).

\title{CrystalPlan: an Experiment Planning Tool for Crystallography}
\shorttitle{CrystalPlan: an Experiment Planning Tool}

     % Authors' names and addresses. Use \cauthor for the main (contact) author.
     % Use \author for all other authors. Use \aff for authors' affiliations.
     % Use lower-case letters in square brackets to link authors to their
     % affiliations; if there is only one affiliation address, remove the [a].

\cauthor[a]{Janik}{Zikovsky}{zikovskyjl@ornl.gov}{address if different from
\aff}
\author[a]{Peter}{Peterson}
\author[a]{Xiaoping}{Wang}
\author[a]{Matthew}{Frost}
\author[a]{Christina}{Hoffmann}

\aff[a]{Spallation Neutron Source, Oak Ridge National Laboratory, P.O. Box 2008
MS-6477, Oak Ridge, TN 37831-6477
\country{USA}}

     % Use \shortauthor to indicate an abbreviated author list for use in
     % running heads (you will need to uncomment it).

\shortauthor{Zikovsky et al.}


     % Keywords (required for Journal of Synchrotron Radiation only)
     % Use the \keyword macro for each word or phrase, e.g. 
     % \keyword{X-ray diffraction}\keyword{muscle}

%\keyword{keyword}


\maketitle                        % DO NOT DELETE THIS LINE

\begin{synopsis}
Describes CrystalPlan, a software program for the automatic creation of 
crystallography experiment plans for time-of-flight diffractometers.
\end{synopsis}

\begin{abstract}
Beam time at large x-ray and neutron scattering facilities is always at a premium.
The CrystalPlan program can calculate the data coverage of a crystal in reciprocal
space in a single-crystal diffraction time-of-flight experiment. CrystalPlan can 
help a user build an experiment plan that will acquire the most data possible, 
with sufficient coverage but limited redundancy, therefore increasing scientific 
productivity. 
An attractive GUI including a 3D viewer and an automated coverage optimizer 
are among its useful features. 
A sample use case of the program with the TOPAZ beamline at SNS will be
presented. 
\end{abstract}



%-------------------------------------------------------------------------
%-------------------------------------------------------------------------
%---------------------------------------------------------------------------------------
\section{Introduction}

The TOPAZ beamline at the Spallation Neutron Source (SNS) at Oak Ridge National Lab (ORNL),
currently finishing commissioning, is a neutron time-of-flight instrument designed to 
acquire single-crystal diffraction data of small-to-moderate sized unit cells with high throughput. 
The final design calls for an array of 48 detectors covering a large fraction of
$2\pi$ steradians; however as of this writing, only 14 of the 48 detectors are
installed. In order to have sufficient data to perform a fit, an experimenter will
typically want to measure $ > 85\%$ of the peaks within a certain
range of d-spacings. However, TOPAZ's complex geometry makes it very difficult to 
come up with an effective set of sample orientations that will
cover all the required peaks without unnecessary redundancy.


Existing software tools [ADD SOME NAMES HERE – see my notes from ACA2010] 
were designed for X-ray experiments where a single incident
wavelength and a large number of sample orientations or sweeps are used to
collect data. These programs cannot be used for TOPAZ, 
since as a time-of-flight instrument, it is able to collect data
from a wide bandwidth of incident neutrons. 
Additionally, this means that relatively few sample orientations are needed in
order to complete a measurement: whereas an X-ray measurement might have
hundreds or thousands of individual frames, a TOPAZ run may have as few as 10.   


The goal of CrystalPlan was to make an easy-to-use software tool to aid users in
planning their experiments. By quickly making an experimental plan that will be
sufficient for the needs of the scientist (enough peaks measured), but without
spending time needlessly measuring the same region in reciprocal space,
CrystalPlan will greatly increase the scientific productivity and throughput of
TOPAZ. More users will be able to measure more samples, faster. 

%---------------------------------------------------------------------------------------
%---------------------------------------------------------------------------------------
%---------------------------------------------------------------------------------------
\section{Architecture}

The program needs to be cross-platform, and run equally well on Linux, Mac OS
and Windows, although Linux is our primary target operating system. Another
design goal was to use of open-source software and toolkits as much as possible. 


The calculation code of CrystalPlan is placed in a model layer and kept
completely separate from eventual GUI elements. This makes it possible to run
calculations using Python scripts, if desired. Unit tests for each part of the
calculations are used to ensure consistent results on different platforms and
systems.    


\subsection{Instrument Configuration}

CrystalPlan supports area detectors with flat faces and arbitrary rectangular
dimensions; this handles the detector types in use at SNS (Anger cameras and
8-pack tube detectors), but the software could easily be extended for more
complicated detector shapes as needed. The detector configuration is loaded from
a text file containing the location, size, orientation and number of pixels of
each detector. 


\subsection{Sample Orientation}

CrystalPlan is designed to be able to use a variety of sample orientation
goniometers. A base Goniometer class is subclassed for specific types of
goniometers, allowing a unified API: in a way, it is a “universal goniometer”,
which uses the conventional goniometer angles of phi, chi, omega as defined in
[REF XXX]. Each class inheriting from Goniometer can convert the desired phi,
chi, omega angles to the internal motor positions it requires. The API allows
for limited goniometers: for example, TOPAZ is currently equipped with a
goniometer where chi is fixed at 45 degrees. CrystalPlan handles this by
limiting the available sample orientation angles accessible to the user.        
 
The data analysis program ISAW is used to calculate the the sample mounting
orientation UB matrix, which is saved to a text file along with the lattice
parameters. CrystalPlan can load this UB matrix file and will use that UB matrix
in coverage calculations.

  
\subsection{Volume coverage}

\subsubsection{Reciprocal Space Representation}

The reciprocal space to be measured is divided into a cubic 3D matrix, the size
of which is determined by the user, who specifies the minimum d-spacing
($d_{min}$, in $\AA$) that he or she is interested in. The resolution of the 3D
matrix is also specified (in $\AA^{-1}$), which determines the number of points to be
modeled. Machine memory limits the resolution that can be achieved.    

Each point of the reciprocal space 3D matrix holds an integer representing the 
number of times that voxel of q-space has been measured – this is the total
coverage matrix. To speed up the calculation of the total, a coverage matrix for
each sample orientation is calculated and saved in memory. This means that the
total coverage matrix can be computed by simply adding multiple 3D matrices
corresponding to each sample orientation. 
 
\subsubsection{Calculating Coverage of One Sample Orientation}


For each sample orientation, a rotation matrix combining the sample's mounting
orientation (the U matrix loaded from ISAW) and the goniometer rotation
($\Phi$, $\Xi$, $\Omega$) is computed. The coverage in reciprocal space is
computed in this way: for each pixel in the face of a detector, the direction of
the scattered beam is known. Then, two incident beam vectors are considered,
corresponding to the low and high wavelength limits of the detectors and/or
source. The scattered beam direction and the two possible incident beam vectors
are used to to find two q vectors, which point in the same direction. At the
high wavelength we find qmin, and the low wavelength limit scattering off of qmax.
Since the instrument has a wide bandwidth, we fill in the 3D q-space matrix by
filling all points from qmin to the  qmax, for all pixels in the face of the
detector. This quickly defines a volume of coverage, as shown in Figure X.  


To reduce memory usage yet allow for quickly trying different detector
configurations, the 3D matrix of each sample orientation is saved a 32- or
64-bit integer, with each bit representing a particular detector. A simple
binary mask can then be applied to quickly compute the coverage if some
detectors are disabled, for example.          




\subsection{Crystal Parameters and Single Reflection Coverage}

Crystal lattice parameters (lengths a,b,c and angles alpha, beta, gamma) are
typed in or loaded from the ISAW UB matrix file. CrystalPlan then generates a
list of HKL peaks with d-spacings larger than the $d_{min}$ specified. For each
sample orientation being simulated and for each peak, a function calculates at
what wavelength lambda and in which direction it scatters. If the wavelength is
within range of the limits of the instrument, the beam direction is projected
onto each detector face to determine where, on each detector surface, it will be
measured. 
 
Each HKL peak is a separate object that holds a list of simulated measurements,
including which detector measured it, at what position on the detector face, at
what wavelength, and for which sample orientation. These data will be used to
graphically display coverage and calculate statistics such as redundancy.            



\subsection{Crystal Symmetry}

When entering the lattice parameters, the crystal's point group symmetry can
also be entered. This information is then used to calculate the increased
measurement redundancy. A list of n 3x3 multiplication matrices is generated
based on the point group; any {hkl} vector can be multiplied by these matrices
to find the n equivalent hkl reflections; for example, for orthorombic (mmm)
symmetry, n=8 and 8 multiplication matrices are generated, which multiply {hkl}
to give {±h, ±k, ±l}.              

For the single reflections, each hkl is tracked separately, but a primary hkl is
found that points to a list of equivalent hkls. The GUI can then display either
all hkl, ignoring crystal symmetry; or combine them to display only part of the
reflections.             

In volume coverage mode, a 4D matrix is generated where each voxel has a list of
n corresponding equivalent voxels. This allows CrystalPlan to quickly compute a
volume coverage map that takes into account crystal symmetry: at each voxel, n
original voxels are checked to see if any of them were measured.         



\subsection{Automatic Coverage Optimizer}

CrystalPlan allows the user to manually enter lists of sample orientations to
simulate; however, finding an optimal plan by trial and error would be very time
consuming. To make the process easier, CrystalPlan includes an Automatic
Coverage Optimizer, which takes as its inputs the desired number of sample
orientations and uses a genetic algorithm to find a solution that satisfies a
user-specified coverage criterion (for example, the user may ask for 85% of
peaks to be measured, or for 90% of q-volume).           

Genetic algorithms have been covered in the literature before, see e.g ref
[INSERT REFERENCE HERE]. In this application, the genes consist of goniometer
angles. For instance, if a goniometer has freedom in 3 angles and the user
requests a 10 different sample orientations, then each individual has 30 genes.
The initial population is created randomly and is made to evolve using mutations
and crossover. For each individual set of sample orientations, the coverage
statistics are computed (taking crystal symmetry into account if desired), and
this is used as the fitness of the given individual in the normal genetic
algorithm process.             

The Automatic Coverage Optimizer can find an acceptable solution (if any is
possible) in times ranging from a few seconds to several minutes, depending on
the difficulty of the problem and the computing hardware. Given that at a
neutron facility, a measurement at a single sample orientation may take a few
hours, this feature can result in significant beam-time savings.           




%---------------------------------------------------------------------------------------
%---------------------------------------------------------------------------------------
%---------------------------------------------------------------------------------------
\section{Implementation}


\subsection{Language and Libraries}

CrystalPlan was written in Python, a powerful, open, cross-platform scripting
language. Python was chosen for its ease of cross-platform deployment, and for
its attractive language features. The numpy and scipy libraries (two
open-source, commonly installed Python libraries) were used for many of the
calculations. Some of the critical calculations were also written with inline C
code, which speed up some calculations by up to $400\times$ as compared with
the pure Python version (which can still be used, in case of incompatibilities).       


\subsection{Graphical User Interface}

\subsubsection{Design}

An attractive and easy-to-use GUI was considered an important part of
CrystalPlan, since the program is meant to be used interactively as a guide to
the experimenter. The user interface was written using wxPython, a mature
cross-platform GUI toolkit, as well as the Enthought Traits GUI. Some of the 2D
plots use matplotlib, and the 3D visualization elements use the Mayavi2 toolkit
for Python, itself based on VTK. All of these libraries are open-source.       

\subsubsection{Main Program Workflow}
Some screenshots of the GUI are shown in Figure Y. The main window consists of a
series of tabs, with workflow proceeding from left to right:  

\begin{itemize}
  \item Q-Space tab: define the volume of interest (minimum d-spacing) and
resolution of reciprocal space. Detectors tab: load instrument detector geometry
files, and enable or disable individual detectors. Goniometer tab: to choose the
goniometer and degrees of orientation freedom of the instrument.
   
  \item Sample tab: enter or load the crystal's lattice parameters and UB matrix.

  \item Try an Orientation tab: interactively rotate a sample and observe the
  changes in coverage in real-time.
   
  \item Add Orientations tab: type in a list of sample orientation angles to add
  to the experiment plan.
 
  \item Experiment Plan tab: a list of the previously calculated sample orientations.
Here you can manage the orientations you will use in the experiment: add,
delete, or temporarily disable them to see the effect on coverage. Once
complete, the plan can be sent to the SNS data acquisition computers in order to
begin a run.
    
\end{itemize} 



\subsubsection{Reciprocal Space Coverage Visualization}
Figures Z -a shows a screenshot of the reciprocal space coverage 3D interface.
The 3D view can represent q-space in two ways: volume coverage, or single-peak
coverage.  

The volume coverage shows a solid isosurface representing the volume measured at
least once. Redundant regions can also be shown using semi-transparent
isosurfaces; or the coverage can be inverted, showing non-measured volumes only;
or sliced between user-selectable $q_{min}$ and $q_{max}$.        

The reflections view (Figure ZZ-a) displays each hkl as a small sphere; the
color of which indicates the number of times the hkl is predicted to be measured
in the current experiment plan.       

Both views can take into crystal symmetry in their display (Figures Z-b and
ZZ-b), using the techniques described previously.   
 
Finally, both views display quick coverage statistics, showing the percentage of
volume (or of the number of peaks) measured, as well as the proportion of
redundancy (voxels or peaks measured 2 or more times).     

\section{Sample Use Case}


\section{Conclusions and Future Development}


Live data processing    …  interface with ISAW … InelasCasetic instruments ??



















     % Appendices appear after the main body of the text. They are prefixed by
     % a single \appendix declaration, and are then structured just like the
     % body text.

%\appendix
%\section{Appendix title}

%Text text text text text text text text text text text text text text
%text text text text text text text.

     %-------------------------------------------------------------------------
     % The back matter of the paper - acknowledgements and references
     %-------------------------------------------------------------------------

     % Acknowledgements come after the appendices

\ack{Acknowledgements}

This research is supported by UT Battelle, LLC under Contract No.
DE-AC05-00OR22725 for the U.S. Department of Energy, Office of Science.



     % References are at the end of the document, between \begin{references}
     % and \end{references} tags. Each reference is in a \reference entry.

\begin{references}
\reference{Author, A. \& Author, B. (1984). \emph{Journal} \textbf{Vol}, 
first page--last page.}
\end{references}

     %-------------------------------------------------------------------------
     % TABLES AND FIGURES SHOULD BE INSERTED AFTER THE MAIN BODY OF THE TEXT
     %-------------------------------------------------------------------------

     % Simple tables should use the tabular environment according to this
     % model

\begin{table}
\caption{Caption to table}
\begin{tabular}{llcr}      % Alignment for each cell: l=left, c=center, r=right
 HEADING    & FOR        & EACH       & COLUMN     \\
\hline
 entry      & entry      & entry      & entry      \\
 entry      & entry      & entry      & entry      \\
 entry      & entry      & entry      & entry      \\
\end{tabular}
\end{table}




     % Postscript figures can be included with multiple figure blocks

\begin{figure}
\caption{Caption describing figure.}
%\includegraphics{fig1.ps}
\end{figure}




\end{document}                    % DO NOT DELETE THIS LINE
%%%%%%%%%%%%%%%%%%%%%%%%%%%%%%%%%%%%%%%%%%%%%%%%%%%%%%%%%%%%%%%%%%%%%%%%%%%%%%
